\documentclass[10pt]{article}

\usepackage[utf8]{inputenc}
\usepackage[french]{babel}
\usepackage[T1]{fontenc}
\usepackage{lmodern}
\usepackage{hyperref}
\usepackage{amsthm}
\usepackage{amscd}
\usepackage{mathtools}
\usepackage{amssymb}
\mathtoolsset{showonlyrefs=true}
\usepackage{fullpage}
\usepackage{graphicx}

%Il faut qu’on écrive une page ou deux par contre, pour expliquer le projet. En gros un titre,   
\title{Ce que l'étude d'impact ne dit pas}

\author{Bruno Scherrer \and Michaël Baudin}


\begin{document}

\maketitle

\section{Résumé}

Le 24 Janvier 2020, le gouvernement a rendu public une 
étude d'impact ayant pour objectif de présenter le projet 
de loi instituant le système universel de retraites. 
L'objectif du présent texte est de permettre de comprendre 
l'influence de cette réforme sur l'équilibre financier macro-économique 
du système de retraite. 
Nous montrons pourquoi les simulations montrent que l'âge de départ à la 
retraite augmente et que le niveau des pensions diminue, contrairement à ce que laisse 
penser l'étude d'impact. 
Ainsi, l'étude d'impact ne présente pas de résultat techniquement faux : 
elle se content de dissimuler l'effet de la réforme \emph{par omission}, 
laissant penser ce qu'elle ne dit, en fait, pas. 

%%%%%%%%%%%%%%%%%%%%%%%%%%%%%%%%%%%%%%%%%%%%%%

\section{Modèle du simulateur officiel du COR}

Dans le but de pouvoir comprendre l'influence des changements indiqués par 
l'étude d'impact, nous souhaiterions pouvoir utiliser le simulateur du COR 
(\url{https://www.cor-retraites.fr/simulateur}). 
Comme nous allons le voir, l'exercice de reproduction des résultats 
de l'étude d'impact révèle les intentions des auteurs de l'étude d'impact. 

Ce simulateur tient compte de deux variables permettant de définir un scénario :
\begin{itemize}
\item le taux de hausse des salaires : +1\%, +1.3\%, +1.5\%, +1.8\%, 
\item le taux de chômage : 4.5%, 7\%, 10\%.
\end{itemize}

Les rapports du COR s'appuient la plupart du temps sur le taux 
de chômage de 7\% et prennent en compte les différents taux de hausse 
des salaires de +1\%, +1.3\%, +1.5\% à +1.8\%. 
Au contraire, l'étude d'impact ne présente généralement qu'une seule 
courbe, correspondant au taux de chômage de 7% avec une hausse des 
salaires de +1.3%. 
Ainsi, on ne peut pas connaître l'influence de ce paramètre sur les calculs 
de l'étude d'impact. 

Une fois le scénario choisi dans le simulateur du COR, 
l'utilisateur doit ajuster trois leviers : 
\begin{itemize}
\item l'âge de départ à la retraite, 
\item le taux de cotisation, 
\item le niveau des pensions par rapport aux salaires. 
\end{itemize}

En sortie, le simulateur du COR calcule :
\begin{itemize}
\item la situation financière du système de retraites, 
\item le niveau de vie des retraités, 
\item la durée de vie passée à la retraite. 
\end{itemize}

On peut utiliser ce simulateur de différentes manières, mais la 
logique qui a dominé dans le passé a consisté à se fixer un objectif 
de niveau de vie des retraités, puis à augmenter l'âge de départ ou 
le taux de cotisations, tout en élevant progressivement le niveau des pensions. 

Reproduire les simulations de l'étude d'impact avec le simulateur du COR 
est donc impossible à priori. 
D'une part, le simulateur ne présente pas le niveau de dépenses du système 
de retraites. 
Or l'objectif du gouvernement est d'abaisser ce niveau de dépenses 
(proche de 14\% en 2020) jusqu'au niveau moyen européen (proche de 12.5\%). 
D'autre part, le simulateur ne permet pas d'imposer l'équilibre financier du système 
de retraites. 
Or cet équilibre financier est l'objet du projet de loi organique. 

C'est pourquoi une inversion mathématique est nécessaire pour pouvoir reproduire 
les résultats de l'étude d'impact. 
C'est la raison pour laquelle nous avons développé un simulateur Open Source 
(\url{https://github.com/brunoscherrer/retraites}) fondé sur les mêmes équations mathématiques 
que le simulateur du COR, mais dont nous avons inversé les relations pour 
pouvoir imposer les paramètres pris en compte dans l'étude d'impact. 

Une difficulté supplémentaire apparaît lorsqu'on fait les calculs : 
l'étude d'impact s'arrête en 2050, alors que les rapports du COR 
se projettent jusqu'en 2070. 
C'est la raison pour laquelle nous sommes contraints de faire des 
hypothèses sur les variables imposées en 2060 et 2070. 

%%%%%%%%%%%%%%%%%%%%%%%%%%%%%%%%%%%%%%%%%%%%%%

\section{Hypothèses de calcul}

Notre calcul se fonde sur trois variables d'entrée :
\begin{itemize}
\item la situation financière du système de retraites, 
\item les dépenses du système de retraites, 
\item l'âge de départ à la retraite. 
\end{itemize}

L'ordre des priorités compte. 
L'équilibre financier prime sur tout le reste. 
Puis vient la diminution des dépenses par rapport à leur niveau actuel. 
L'âge de départ à la retraite doit donc augmenter. 
Dans la suite du texte, nous allons préciser quantitativement 
les évolutions prévues de chaque paramètre. 

%%%%%%%%%%%%%%%%%%%%%%%%%%%%%%%%%%%%%%%%%%%%%%

\subsection{L'équilibre financier}

L'équilibre financier est certainement la variable 
la plus facile à ajuster. 

L'étude d'impact, page 180, présente une analyse du solde financier du système de retraite 
avant et après réforme.
Le graphique 63 suivant est extrait de l'étude d'impact. 

\begin{center}
\includegraphics[width=0.9\textwidth]{EtudeImpact-situation-financiere.png}
\end{center}

Le texte indique : "Compte tenu des hypothèses décrites plus haut 
et en y ajoutant une mesure conventionnelle de redressement à court 
terme afin d’être à l’équilibre en 2027, le graphique ci-après présente 
la trajectoire du solde du SUR en la comparant à la trajectoire 
contrefactuelle (hors réforme) à l’horizon 2050."

C'est pourquoi nous devons considérer un solde financier :
\begin{itemize}
\item inchangé avant 2020,
\item linéairement croissant jusqu'à un solde nul en 2027,
\item puis nul ensuite.
\end{itemize}

Nous avons imposé ce solde financier dans notre propre simulateur. 
Les résultats que nous obtenons sont les suivants. 

\begin{center}
\includegraphics[width=0.7\textwidth]{Simulation-Solde-Financier.png}

\includegraphics[width=0.5\textwidth]{Simulation-legende.png}
\end{center}

La légende de notre simulation insiste sur le fait que 
le système doit être à l'équilibre financier \emph{quelque soit la 
conjoncture économique}. 

%%%%%%%%%%%%%%%%%%%%%%%%%%%%%%%%%%%%%%%%%%%%%%

\subsection{Les dépenses de retraite}

L'étude d'impact, page 174, présente une analyse du niveau de dépenses 
en \% de PIB : "Ce taux est plus élevé que ce qu’on observe dans 
les autres pays européens. Les prestations de vieillesse-survie 
(correspondant au champ comparable internationalement, plus 
large que les dépenses du seul système de retraite) représentent 
14,4 \% du PIB en France, contre 12,6 \% du PIB dans l’UE-15 et 12,3 \% dans l’UE-28."
On comprend donc que l'objectif est de se rapprocher de la moyenne européenne. 

Dans l'étude d'impact, page 174, le graphique 58 présente les dépenses constatées 
et projetées du système de retraite actuel, en points de PIB.

\begin{center}
\includegraphics[width=0.9\textwidth]{EtudeImpact-depenses-constatees-projetees.png}
\end{center}

Pour cette trajectoire de dépenses, nous considérons les mêmes 
niveaux de dépenses de les dépenses de l'étude d'impact de 2020 à 2050. 
Pour la période 2050-2070, nous faisons l'hypothèse que le niveau 
de dépense s'abaisse jusqu'à la moyenne 12,6 \% du PIB dans l’UE-15. 
La figure suivante présente le résultat. 

\begin{center}
\includegraphics[width=0.5\textwidth]{Simulation-Depenses.png}

\includegraphics[width=0.5\textwidth]{Simulation-legende.png}
\end{center}

%%%%%%%%%%%%%%%%%%%%%%%%%%%%%%%%%%%%%%%%%%%%%%

\subsection{L'âge de départ en retraite}

Reproduire l'âge de départ en retraite dans notre simulation 
pose des difficultés. 

Le graphique 49 page 139 de l'étude d'impact de Janvier 2020 est présenté ci-dessous.

\begin{center}
\includegraphics[width=0.9\textwidth]{EtudeImpact-AgeDepartRetraite.png}
\end{center}

Le texte précise : "Au total, en tenant compte de l’ensemble de ces 
décalages, l’âge moyen de départ serait plus élevé dans le système 
universel : 64 ans et 5 mois contre 64 ans et 10 mois environ dans 
le système actuel pour la génération 1990."

Sur le graphique de l'étude d'impact, nous lisons les valeurs numériques suivantes après réforme :
\begin{itemize}
\item un âge de départ à 63,8 ans pour la génération 1975,
\item un âge de départ à 64,83 ans pour la génération 1990.
\end{itemize}

L'âge de départ pour la génération 1975 est lu sur le graphique. 
L'âge de départ pour la génération 1990 se déduit du texte, qui 
indique un âge de départ de "64 ans et 10 mois". 
Puisque 10/12 = 0.83, nous en déduisons un âge de départ égal à 64.83 pour 
la génération 1990. 

On observe que la génération 1975 partira en retraite en 2039 dans le scénario de l'étude d'impact puisque 1975 + 63.8 = 2038.8.
Remarquons que l'horizon temporel de l'étude d'impact se projette dans le graphique jusqu'à l'année 2054, puisque 1990 + 64.83 = 2054.83. 


Pour calculer l'âge de départ en retraite en fonction de l'année du départ, 
nous réalisons une inversion mathématique.  

Notre calcul se décompose en trois parties :
\begin{itemize}
\item jusqu'à l'année 2039, nous utilisons les données du COR, 
\item de 2039 à 2055, nous utilisons les données de l'étude d'impact,
\item de 2055 à 2070, nous extrapolons. 
\end{itemize}

Le résultat de notre simulation est donc le suivant. 

\begin{center}
\includegraphics[width=0.5\textwidth]{Simulation-Age.png}

\includegraphics[width=0.5\textwidth]{Simulation-legende.png}
\end{center}

On observe que l'âge de départ à la retraite est donc 
significativement supérieur dans l'étude d'impact. 
Notre extrapolation a mené à un âge de départ à la retraite égal à 
environ 66 ans en 2070. 
Nous ne savons pas si cet âge est réaliste, mais nous notons deux éléments. 
\begin{itemize}
\item Le COR prévoyait une augmentation de l'âge de départ moins 
forte à partir de 2040. 
\item Si l'âge réel de départ à la retraite ne suit pas la courbe 
que nous avons imposée, alors les pensions de retraites que nous 
obtiendrons en conséquence seront inférieures à celles que nous avons 
simulées. 
\end{itemize}

%%%%%%%%%%%%%%%%%%%%%%%%%%%%%%%%%%%%%%%%%%%%%%

\section{Résultats}

%%%%%%%%%%%%%%%%%%%%%%%%%%%%%%%%%%%%%%%%%%%%%%

\subsection{Le niveau de vie des retraités par rapport à l'ensemble de la population}

Le graphique suivant présente le niveau de vie des retraités par rapport aux 
actifs. 
D'après la documentation technique du COR : "Le niveau de vie est défini 
par l’INSEE, au niveau de chaque ménage, comme le revenu disponible (tenant
compte de l’ensemble des ressources et après prélèvements et transferts sociaux) 
divisé par le nombre d’unités de consommation dans le ménage 
(qui dépend du nombre de personnes qui le composent, cf. page 4)."

\begin{center}
\includegraphics[width=0.5\textwidth]{Simulation-RNV.png}

\includegraphics[width=0.5\textwidth]{Simulation-legende.png}
\end{center}

On observe que le niveau de vie des retraités 
baisse de 105\% en 2020 jusqu'en 2050 entre 85\% et 90\%, puis se stabilise ensuite. 

%%%%%%%%%%%%%%%%%%%%%%%%%%%%%%%%%%%%%%%%%%%%%%

\subsection{Le niveau des pensions par rapport aux actifs}

Le graphique suivant présente le rapport entre la pension moyenne et le salaire moyen. 

\begin{center}
\includegraphics[width=0.5\textwidth]{Simulation-P.png}

\includegraphics[width=0.5\textwidth]{Simulation-legende.png}
\end{center}

On observe que le niveau des pensions par rapport aux actifs 
baisse de 50\% en 2020 jusqu'en 2050 autour de 39\%, puis se stabilise ensuite. 
Relativement au niveau actuel de ce ratio, la baisse est donc d'environ 20\%. 

%%%%%%%%%%%%%%%%%%%%%%%%%%%%%%%%%%%%%%%%%%%%%%

\subsection{Le niveau des pensions dans l'étude d'impact}

Le lecteur de l'étude d'impact sera très étonné à la lecture d'un 
tel graphique. 
En effet, à la page 176 de l'étude d'impact, le graphique 59 présente 
une pension annuelle moyenne plutôt favorable dans le système 
universel par rapport à la situation hors réforme. 
Le texte précise : "En moyenne, les niveaux des pensions servies augmentent 
avec la mise en place du système universel."

\begin{center}
\includegraphics[width=0.7\textwidth]{EtudeImpact-PensionAnnuelle.png}
\end{center}

L'étude d'impact ne ment pas. 
Notre simulation ne ment pas plus. 
Qui dit vrai ?

Pour comprendre la situation, on peut consulter le rapport du COR de Juin 2019 qui 
présente l'évolution de la pension moyenne de l'ensemble des retraités relative 
au revenu d'activité moyen. 
En utilisant cet indicateur, le COR compare le revenu des retraités aux revenu des 
actifs. 

\begin{center}
\includegraphics[width=0.5\textwidth]{COR-Juin-2019-P.png}
\end{center}

On observe que, dans le système actuel, la pension moyenne relative au revenu d'activité 
moyen baisse jusqu'à atteindre entre 33\% et 39\% en fonction du scénario. 
Cette réalité n'est pas une conséquence du système de retraite tel qu'il était conçu 
à l'origine. 
C'est une conséquence des différentes réformes qui ont étées menées 
à partir des années 1990 et suivante. 

C'est pourquoi l'étude d'impact peu montrer une pension annuelle moyenne 
dans le système universel supérieure à la pension hors réforme : 
les pensions baissent moins que ce qui était prévu, mais elles baissent 
tout de même relativement aux salaires ! 
Ainsi, en changeant d'indicateur, l'étude d'impact peut montrer une situation 
dont l'apparence est favorable. 

%%%%%%%%%%%%%%%%%%%%%%%%%%%%%%%%%%%%%%%%%%%%%%

\section{Conclusion}

Nous avons vu comment la logique du projet de loi est une 
rupture dans le pilotage du système de retraites, 
imposant l'équilibre financier et le volume des dépenses 
à priori : les pensions devront donc s'ajuster \emph{en conséquence}. 
De plus, nous avons observé ce qui nous était caché dans l'étude 
d'impact, c'est à dire ce qui se passe entre 2050 et 2070. 
Enfin, nous avons vu comment la réforme est fallacieusement 
montrée comme avantageuse pour des niveaux de pensions 
objectivement désavantageux par rapport à la situation 
actuelle. 

Les conséquences de la réforme sont simples : 
si cette loi est votée, l'âge de départ à la retraite 
devra augmenter et le niveau de pension par rapport aux actifs va 
baisser de -20\% relativement au niveau actuel. 
Le niveau de vie des futurs retraités par rapport à l'ensemble de la 
population sera très inférieur : prendre sa retraite sera synonyme de 
déclassement. 
Dans la situation où les futurs retraités auraient la fantaisie 
de prendre leur retraite avant l'âge d'équilibre, par choix ou par contrainte, 
la situation sera pire encore. 

Les parlementaires qui auraient pris le soin de lire l'impressionnante 
étude d'impact (1029 pages !) en sont pour leur frais. 
En effet, le texte ne présente pas les informations essentielles qui leur auraient étés 
utiles pour s'informer sur les conséquences de la réforme. 
Pire : les informations présentées dans l'étude détournent l'attention de l'essentiel 
mettant en avant des détails techniques accessoires et affichant fallacieusement 
des informations soigneusement choisies. 

Les citoyens que nous sommes peuvent comprendre qu'une proposition de loi 
aille dans un sens politique que nous ne partageons pas : c'est la 
liberté de la vie élective. 
En revanche, nous ne pouvons pas accepter que la décision politique soit 
prise de données qui ne sont pas ouvertes, 
de calculs qui ne sont pas publics et, finalement, sur la base d'études trompeuses. 

%%%%%%%%%%%%%%%%%%%%%%%%%%%%%%%%%%%%%%%%%%%%%%

\section{Annexe}

Pour les lecteurs désirant reproduire les simulations de ce texte, 
nous présentons ci-dessous les paramètres que nous avons calculés.
\begin{verbatim}
Scenario : Hausse des salaires: +1,3%/an, Taux de chômage: 7%
Annéee, Age,      Cotis., Pension:
 2020 : 62.2 ans, 30.8 %, 50.2 %
 2025 : 62.6 ans, 31.4 %, 49.1 %
 2030 : 63.0 ans, 31.3 %, 46.8 %
 2040 : 63.9 ans, 30.9 %, 43.4 %
 2050 : 64.5 ans, 29.5 %, 40.1 %
 2060 : 65.2 ans, 28.8 %, 39.0 %
 2070 : 65.8 ans, 28.6 %, 38.8 %
\end{verbatim}

Pour tester graphiquement les effets de ces paramètres, nous vous recommandons le 
simulateur du collectifs "Nos retraites" \url{https://nosretraites.github.io/roc-retraites}. 

%%%%%%%%%%%%%%%%%%%%%%%%%%%%%%%%%%%%%%%%%%%%%%

\section{Références}

\begin{itemize}
\item Projet de loi instituant un système universel de retraite. - Etude d'impact, 24 janvier 2020
\item Évolutions et perspectives des retraites en France. Rapport annuel du COR – Juin 2019
\item \url{https://www.cor-retraites.fr/simulateur}
\item Simulateur du COR / Documentation technique - juillet 2016
\item \url{https://github.com/brunoscherrer/retraites}
\item \url{https://nosretraites.github.io/roc-retraites/}
\end{itemize}

\end{document}

